\documentclass[a4paper,12pt]{article}
\usepackage[utf8]{inputenc}
\usepackage[T2A]{fontenc}
\usepackage[russian]{babel}
\usepackage{geometry}
\usepackage{graphicx}
\usepackage{hyperref}
\usepackage{array}
\usepackage{tikz}
\usepackage{caption}
\usepackage{float}
\usepackage{needspace} 

\geometry{margin=2.5cm}

\usepackage{listingsutf8}
\lstset{
  inputencoding=utf8,
  basicstyle=\ttfamily\small,
  breaklines=true,
  frame=single
}

\title{Формат ARC/DIR и утилита \texttt{arcdir\_tool}\\Полная техническая документация}
\author{}
\date{\today}

\begin{document}
\maketitle
\tableofcontents
\newpage

\section{Введение}

Настоящий документ представляет систематизированное техническое описание парного архивного формата \texttt{.arc}/\texttt{.dir} 
и утилиты \texttt{arcdir\_tool}, предназначенной для извлечения, упаковки и управления архивными данными. 
Описаны синтаксис командной строки, алгоритмы обработки данных, структура бинарных файлов, правила выравнивания и практические примеры применения.  

Утилита реализована на языке C, обеспечивая:  
\needspace{3\baselineskip}
\begin{itemize}
  \item Высокую производительность благодаря прямому доступу к файлам;
  \item Кроссплатформенную совместимость (Windows и POSIX-системы);
  \item Детерминированное поведение при одинаковых входных данных.
\end{itemize}

\section{Общая концепция}

Формат архива состоит из двух взаимосвязанных файлов:  
\needspace{3\baselineskip}
\begin{itemize}
  \item \texttt{.arc} --- последовательная конкатенация сырых данных файлов;
  \item \texttt{.dir} --- индекс с метаданными, включающими смещения, размеры и пути файлов.
\end{itemize}

Файл \texttt{.dir} функционирует как структурированный индекс для \texttt{.arc}. Каждая запись сопоставляет путь к файлу с его смещением и размером в ARC-файле.

\section{Архитектура системы}

\begin{table}[H]
\centering
\begin{tikzpicture}[node distance=2.5cm, auto]
  \node[draw, rectangle] (user) {Юзер};
  \node[draw, rectangle, right of=user] (tool) {arcdir\_tool};
  \node[draw, rectangle, right of=tool, yshift=1cm] (arc) {файл .arc};
  \node[draw, rectangle, right of=tool, yshift=-1cm] (dir) {файл .dir};

  \draw[->] (user) -- (tool);
  \draw[->] (tool) -- (arc);
  \draw[->] (tool) -- (dir);
\end{tikzpicture}
\caption{Общая архитектура утилиты \texttt{arcdir\_tool}}
\end{table}
\newpage

\section{Последовательность выполнения программы}

Процесс выполнения утилиты \texttt{arcdir\_tool} описывается как линейная последовательность действий:

\needspace{15\baselineskip}
\begin{enumerate}
  \item Разбор аргументов командной строки и идентификация режима работы: \texttt{EXTRACT}, \texttt{PACK} или \texttt{PACK\_BIN}.
  \item \textbf{Режим EXTRACT:}
    \begin{itemize}
      \item Открытие файлов ARC и DIR в режиме чтения;
      \item Последовательное чтение записей из файла DIR;
      \item Применение опциональных правил перенаправления путей;
      \item Запись данных файлов в соответствующие выходные директории.
    \end{itemize}
  \item \textbf{Режимы PACK и PACK\_BIN:}
    \begin{itemize}
      \item Рекурсивное сканирование указанных директорий и файлов;
      \item Применение фильтра расширений (\texttt{.bin} для режима PACK\_BIN);
      \item Лексикографическая сортировка всех обнаруженных путей;
      \item Запись индекса DIR и последовательное размещение данных файлов в ARC;
      \item Применение дополнений для выравнивания строк пути (кратность 4 байтам) и данных файлов (кратность 32 байтам).
    \end{itemize}
  \item Закрытие всех файлов и корректное завершение работы.
\end{enumerate}

\section{Интерфейс командной строки}

\subsection{Синтаксис}

\begin{lstlisting}
arcdir_tool EXTRACT <archive.arc> <index.dir> [<src> <dst>]...
arcdir_tool PACK <archive.arc> <index.dir> [path...]
arcdir_tool PACK_BIN <archive.arc> <index.dir> [path...]
\end{lstlisting}

\subsection{Режимы}

\subsubsection{EXTRACT}

\needspace{4\baselineskip}
\begin{itemize}
  \item Извлечение файлов из существующего архива \texttt{.arc}/\texttt{.dir}.
  \item При отсутствии пар \texttt{<src> <dst>} извлекаются все файлы по исходным путям.
  \item При указании пар извлекаются только соответствующие файлы и записываются в целевые директории.
\end{itemize}

\subsubsection{PACK}

\needspace{5\baselineskip}
\begin{itemize}
  \item Создание новой пары архивных файлов \texttt{.arc}/\texttt{.dir}.
  \item Аргументы могут быть как файлами, так и директориями;
  \item Директории обрабатываются рекурсивно;
  \item Все обнаруженные файлы включаются в алфавитном порядке.
\end{itemize}

\subsubsection{PACK\_BIN}

\needspace{3\baselineskip}
\begin{itemize}
  \item Идентичен режиму PACK, но рекурсивно найденные файлы фильтруются по расширению \texttt{.bin};
  \item Явно указанные файлы включаются независимо от расширения.
\end{itemize}

\subsection{Примеры использования}

\begin{lstlisting}
# Extract all files
arcdir_tool EXTRACT data.arc data.dir

# Extract a single file to a custom path
arcdir_tool EXTRACT data.arc data.dir a/b.bin out.bin

# Pack a directory recursively
arcdir_tool PACK data.arc data.dir assets/

# Pack only .bin files from directories
arcdir_tool PACK_BIN data.arc data.dir assets/
\end{lstlisting}


\section{Обнаружение и сортировка файлов}

Процедура упаковки файлов включает следующие этапы:
\needspace{7\baselineskip}
\begin{enumerate}
  \item Анализ каждого аргумента командной строки;
  \item Рекурсивное сканирование директорий;
  \item Непосредственное добавление файлов;
  \item Применение опционального фильтра по расширению;
  \item Лексикографическая сортировка результирующего списка файлов.
\end{enumerate}

\section{Структура бинарного формата}

Все числовые поля представлены в формате big-endian.

\subsection{Файл DIR}

\subsubsection{Заголовок}

\begin{table}[H]
\centering
\begin{tabular}{|l|l|l|}
\hline
Смещение & Размер & Описание \\
\hline
0x00 & 4 байта & Общий размер файла DIR (u32, big-endian) \\
0x04 & 4 байта & Количество записей (u32, big-endian) \\
\hline
\end{tabular}
\caption{Структура заголовка файла DIR}
\end{table}

\subsubsection{Запись индекса}

Каждая запись имеет структуру:

\begin{table}[H]
\centering
\begin{tabular}{|l|l|l|}
\hline
Поле & Размер & Описание \\
\hline
Offset & 4 байта & Смещение начала данных файла в ARC (u32, big-endian) \\
Size & 4 байта & Размер данных файла в байтах (u32, big-endian) \\
PathLen & 4 байта & Длина строки пути с учётом выравнивания (u32, big-endian) \\
Path & PathLen байт & Нуль-терминированная строка пути, выровненная до 4 байт \\
\hline
\end{tabular}
\caption{Структура записи файла DIR}
\end{table}

\subsection{Файл ARC}

\needspace{3\baselineskip}
\begin{itemize}
  \item Последовательная запись сырых данных файлов;
  \item Добавление выравнивающего дополнения после каждого файла для обеспечения кратности 32 байтам;
  \item Дополнение заполняется байтами \texttt{0xCC}.
\end{itemize}

\subsection{Правила выравнивания}

\subsubsection{Выравнивание строк пути}

\begin{lstlisting}[language=C]
pad = (~(len - 1)) & (4 - 1);
total_len = len + pad;
\end{lstlisting}

\subsubsection{Выравнивание данных файлов}

\begin{lstlisting}[language=C]
pad = (~(data_size - 1)) & (32 - 1);
\end{lstlisting}
\newpage

\section{Алгоритм упаковки}

\needspace{12\baselineskip}
\begin{enumerate}
  \item Открытие ARC и DIR в режиме записи;
  \item Резервирование 8 байт в DIR для заголовка;
  \item Для каждого отсортированного файла:
    \begin{itemize}
      \item Считывание содержимого файла в оперативную память;
      \item Формирование записи DIR: смещение в ARC, размер файла, длина пути с выравниванием, строка пути с выравниванием;
      \item Последовательная запись данных файла в ARC;
      \item Применение выравнивающего дополнения до кратности 32 байтам с использованием 0xCC.
    \end{itemize}
  \item Запись общего размера файла DIR и количества записей в начале файла.
\end{enumerate}

\section{Алгоритм извлечения}

\needspace{12\baselineskip}
\begin{enumerate}
  \item Открытие ARC и DIR в режиме чтения;
  \item Чтение заголовка DIR (общий размер и количество записей);
  \item Для каждой записи:
    \begin{itemize}
      \item Чтение смещения, размера и длины пути;
      \item Чтение строки пути и удаление выравнивающего дополнения;
      \item Применение опциональных правил перенаправления;
      \item Переход к смещению в ARC;
      \item Чтение данных файла и запись в целевую директорию.
    \end{itemize}
\end{enumerate}

\section{Обработка путей}

\needspace{3\baselineskip}
\begin{itemize}
  \item Пути представлены как байтовые строки, совместимые с кодировкой Shift-JIS;
  \item Разделители директорий нормализуются к символу '/' при упаковке;
  \item Таблицы перенаправления применяются опционально во время извлечения.
\end{itemize}

\section{Обработка ошибок}

\needspace{3\baselineskip}
\begin{itemize}
  \item Чтение фиксированного количества байт; обнаружение EOF вызывает ошибку;
  \item Размер заголовка DIR используется для проверки корректности;
  \item Ошибки ввода/вывода приводят к немедленному завершению работы программы.
\end{itemize}

\section{Детерминированность}

Для идентичных входных данных:
\needspace{3\baselineskip}
\begin{itemize}
  \item Структура записей DIR и размещение данных в ARC воспроизводимы;
  \item Лексикографическая сортировка обеспечивает детерминированность;
  \item Фиксированные правила выравнивания сохраняют корректное смещение данных.
\end{itemize}

\section{Заключение}

Формат ARC/DIR представляет собой индексированный контейнер данных:  
\needspace{4\baselineskip}
\begin{itemize}
  \item \texttt{.arc} содержит последовательные блоки данных;
  \item \texttt{.dir} содержит метаданные, сопоставляющие пути файлов со смещениями и размерами;
  \item Используются целые числа в формате big-endian и строгие правила выравнивания;
  \item Утилита \texttt{arcdir\_tool} обеспечивает полное управление упаковкой и извлечением архивов с детерминированным результатом.
\end{itemize}

\end{document}
